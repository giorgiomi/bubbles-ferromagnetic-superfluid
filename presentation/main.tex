%Use one of the two documentclass lines depending on aspect ratio needed
% for 4x3 aspect ratio slides
%\documentclass{beamer}
%for 16x9 (modern wide screen) aspect ratio slides
\documentclass[aspectratio=169]{beamer}
\usepackage[T1]{fontenc}
\usepackage[utf8]{inputenc}
\usepackage{tikz-cd,wrapfig}
\usepackage{tcolorbox}
\usepackage{listings}
\usepackage[export]{adjustbox}

% Maths
\usepackage{amsmath}
\usepackage{amssymb}
\usepackage{physics}
\usepackage{siunitx}

% Backup slides
\newcommand{\backupbegin}{
    \newcounter{finalframe}
    \setcounter{finalframe}{\value{framenumber}}
}
\newcommand{\backupend}{
    \setcounter{framenumber}{\value{finalframe}}
}

% Oxford Maths theming
\usetheme{unitnphysics}

% Set author etc info
\title[Bubbles in a ferromagnetic superfluid] %short version of title for slide footer
{Bubbles in a ferromagnetic superfluid} %full title for titlepage
\author{\textbf{Candidate}: Giorgio Micaglio\\\textbf{Supervisor}: dr.\ Alessandro Zenesini}
\institute{Bachelor's Degree in Physics}
\date[March 10, 2025]  %short date for slide footer
{March 10, 2025} %main date for title page,
                        %can overload it to show say 'Conference X, Date Y'


%% Now for the actual slides %%
\begin{document}

\begin{frame}[plain]
  \titlepage
\end{frame}

\begin{frame}{\textbf{Overview}}
  \begin{columns}
      \begin{column}{0.7\textwidth}
          This presentation will cover:
          \begin{itemize}
              \item \textbf{Introduction}
              \item \textbf{Theoretical background}: Ferromagnetism in coherently coupled two-component spin-mixtures
              \item \textbf{Data analysis}: Characterization of false vacuum decay bubbles
              \item \textbf{Conclusions}
          \end{itemize}
      \end{column}
      \begin{column}{0.3\textwidth}
          \begin{figure}
              \centering
              \includegraphics[width=\linewidth]{../thesis/figures/chap1/artistic.png}
          \end{figure}
      \end{column}
  \end{columns}
\end{frame}

\begin{frame}{\textbf{Introduction}}
  Why \textbf{bubbles} in a ferromagnetic superfluid?

  ~

  \begin{itemize}
      \item First \textbf{experimental observation} of false vacuum decay (FVD) in the Pitaevskii BEC Center laboratories of the University of Trento.
      \item FVD provides information on \textbf{metastability} and is studied from quantum systems to cosmology
      \item Framework: \textbf{quantum gas} of $^{23}$Na atoms optically trapped and cooled below the condensation temperature
  \end{itemize}
  
\end{frame}

% \section{\textbf{Theoretical background}}

% \begin{frame}{\textbf{Theoretical background}: Ideal Bose gas}
%   The ideal Bose gas is a quantum system of $N$ non-interacting bosons, described by statistical mechanics.
%   \begin{equation*}
%       \langle n_i \rangle = \frac{1}{e^{\beta(\epsilon_i - \mu)}-1}
%   \end{equation*}
%   \onslide<2->
%   The occupation number of the ground state $N_0 = \langle n_0\rangle$ corresponds to the condensation. There is a phase transition at $T = T_c$.
%   \begin{equation*}
%       \frac{N_0}{N} = 1-\left(\frac{T}{T_c}\right)^\alpha \quad \text{for } T < T_c
%   \end{equation*}
%   In a finite box $\alpha = 3/2$, in harmonic confinement $\alpha = 3$.
% \end{frame}

\begin{frame}{\textbf{Theoretical background}: Gross-Pitaevskii equation}
  A system of \textbf{weakly-interacting bosons} can be described by a mean-field approximation with a single wavefunction, yielding the GPE:
  \begin{equation*}
      i\hbar \pdv{\psi(x,t)}{t} = \left[ 
          -\frac{\hbar^2}{2m}\nabla^2 + V(x,t) + g|\psi(x,t)|^2
      \right] \psi(x,t)
  \end{equation*}
  \pause
  In the stationary case:
  \begin{equation*}
      \left[ 
          -\frac{\hbar^2}{2m}\nabla^2 + V(x) + g|\psi(x)|^2
      \right] \psi(x) = \mu \psi(x)
  \end{equation*}
  When the interaction dominates on the kinetic term:
  \begin{equation*}
      n(x) = \frac{\mu - V(x)}{g} \quad \Rightarrow \quad R_{\rm TF} = \sqrt{\frac{2\mu}{m\omega^2}}
  \end{equation*}
\end{frame}

\begin{frame}{\textbf{Theoretical background}: Two-component mixtures}
  In a two-component gas, the GPEs are \textbf{coupled} because of the inter-species interaction constant:
  \begin{align*}
      &\left[ -\frac{\hbar^2}{2m_a}\nabla^2 + V(x) + g_{aa}|\psi_a(x)|^2 + g_{ab}|\psi_b(x)|^2
      \right] \psi_a(x) = \mu_a \psi_a(x) \\
      &\left[ -\frac{\hbar^2}{2m_b}\nabla^2 + V(x) + g_{ab}|\psi_a(x)|^2 + g_{bb}|\psi_b(x)|^2
      \right] \psi_b(x) = \mu_b \psi_b(x)
  \end{align*}
  Depending on the values of $g_{aa}$, $g_{bb}$ and $g_{ab}$, the \textbf{ground state} of the system can behave in different manners
\end{frame}

% \begin{frame}{\textbf{Theoretical background}: Two-component mixtures}
%     The mixture can be miscible or immiscible: buoyancy effect
%     \begin{figure}
%         \centering
%         \includegraphics[width=0.7\textwidth]{../thesis/figures/chap1/buoyancy.png}
%     \end{figure}
% \end{frame}

\begin{frame}{\textbf{Theoretical background}: Coherent coupling}
  When species of the same atom, \textbf{coupling radiation} between $\ket{a}$ and $\ket{b}$:
  \[
      \hbar\Omega_R\exp{-i\omega_{\rm cpl}t+\phi} \quad \text{with} \quad \omega_{\rm cpl} = \omega_{ab} + \delta_B
  \]
  Two distinct \textbf{energy channels}: spin $nZ = n_a-n_b$ and total density $n = n_a+n_b$
  \pause

  ~

  Double-well energy landscape
  \begin{equation*}
      E_{\rm MF}(Z) = -\hbar\left(|\delta g|n Z^2 + 2\Omega_R \sqrt{1-Z^2} + 2\delta_{\rm eff} Z\right)
      \label{eq:E_MF}
  \end{equation*}

  The order parameter is the ratio $\dfrac{|\delta g|n}{\hbar\Omega_R}$, the effective detuning is $\delta_{\rm eff} = \delta_B - n(g_{aa}-g_{bb})$
\end{frame}

\begin{frame}{\textbf{Theoretical background}: Magnetic model}
  \begin{minipage}{0.7\textwidth}
    \begin{figure}
      \centering
      \includegraphics[width=\linewidth]{../thesis/figures/chap1/ferro.png}
    \end{figure} 
  \end{minipage}
  \hspace{0.01\textwidth}
  \begin{minipage}{0.27\textwidth}
    \begin{align*}
      &\alpha\to\delta g\\
      &B_3\to\delta_{\rm eff}\\
      &B_1\to\Omega_R
    \end{align*}
  \end{minipage}
\end{frame} 

\begin{frame}{\textbf{Theoretical background}: False Vacuum Decay}
  \begin{minipage}{0.32\textwidth}
    \begin{itemize}
      \item Quantum tunnelling from A to B (stochastic)
      \item Decay from B to C
      \item Problem: when to take the shot?
    \end{itemize}
  \end{minipage}
  \hspace{0.01\textwidth}
  \begin{minipage}{0.65\textwidth}
    \begin{figure}
      \centering
      \includegraphics[width=\linewidth]{../thesis/figures/chap1/FVD.png}
    \end{figure} 
  \end{minipage}
\end{frame}

\begin{frame}{\textbf{Data analysis}: Experimental platform}
  \begin{itemize}
    \item $^{23}$Na atoms prepared in the state $\ket{F,m_F} = \ket{2, -2} = \ket{\uparrow}$, which is coupled to the state $\ket{1, -1} = \ket{\downarrow}$
    \item \textbf{Harmonic trapping} potential with $\omega_\perp/2\pi = 2$ \unit{\kilo\hertz} and $\omega_x/2\pi = 20$ \unit{\kilo\hertz}
    \item Thomas-Fermi radii $R_\perp = 2\ \unit{\micro\meter}$ and $R_x = 200\ \unit{\micro\meter}$ (cigar-shaped)
    \item Reduction to \textbf{1D system} and spin-selective imaging yield the densities $n_\uparrow(x)$, $n_\downarrow(x)$ and the magnetization $Z(x)$
  \end{itemize}  
  \pause
  \begin{figure}
    \centering
    \includegraphics[width=0.8\textwidth]{../thesis/figures/chap2/two_bubbles.png}
  \end{figure}
\end{frame}

\begin{frame}{\textbf{Data analysis}: Bubble fits}
  % \vspace{-0.1cm}
  % \begin{figure}
  %     \centering
  %     \includegraphics[width=0.7\textwidth]{../thesis/figures/chap2/two_bubbles.png}
  % \end{figure}
  % \vspace{-0.5cm}
  \begin{figure}
      \centering
      \includegraphics[width=\textwidth]{../thesis/figures/chap2/arctan_fit.png}
  \end{figure}
\end{frame}

\begin{frame}{\textbf{Data analysis}: Shot sorting}
  \begin{figure}
      \centering
      \includegraphics[width=0.9\textwidth]{../thesis/figures/chap2/shot_sorting.png}
  \end{figure}
\end{frame}

\begin{frame}{\textbf{Data analysis}: Parameters clustering}
  \begin{minipage}{0.70\textwidth}
    % \vspace{-0.05cm}
    \begin{figure}
        \centering
        \includegraphics[width=\linewidth]{../thesis/figures/chap2/b_param_cluster.png}
    \end{figure}
  \end{minipage}
  \hspace{0.01\textwidth}
  \begin{minipage}{0.27\textwidth}
    Fit function:
    \[
      \sigma_B(t) = A\left(\frac{t}{1\ \unit{\milli\second}}\right)^B
    \]  
  \end{minipage}
\end{frame}

\begin{frame}{\textbf{Data analysis}: FFT and ACF}
  % \vspace{-0.2cm}
  \begin{figure}
      \centering
      \includegraphics[width=0.7\textwidth]{../thesis/figures/chap2/inside_omdet.png}
  \end{figure}
\end{frame}

\begin{frame}{\textbf{Data analysis}: FFT and ACF}
  \begin{figure}
      \centering
      \includegraphics[width=0.8\textwidth]{../thesis/figures/chap2/inside_fft_avg.png}
  \end{figure}
\end{frame}

\begin{frame}{\textbf{Data analysis}: ACF inside (fits)}
  \hspace{-0.05\textwidth}
  \begin{minipage}{0.47 \textwidth}
    \centering
    \adjincludegraphics[width=\linewidth, trim={0 0 {.5\width} 0}, clip]{../thesis/figures/chap2/fit_size_inside.png}
  \end{minipage}
  \hspace{0.01\textwidth}
  \begin{minipage}{0.50 \textwidth}
      Fit function:
      \[
        \mathcal{A}_{\rm fit}(x) = (1 - \Delta)\cos(\frac{\pi x}{\ell_2})\exp[-\frac{1}{2}\left(\frac{x}{\ell_1}\right)^\alpha] + \Delta
      \]
  \end{minipage}
\end{frame}

\begin{frame}{\textbf{Data analysis}: ACF inside (parameters)}
  \begin{figure}
      \centering
      \includegraphics[width=\textwidth]{../thesis/figures/chap2/param_size_inside.png}
  \end{figure}
\end{frame}

\begin{frame}{\textbf{Data analysis}: ACF inside (parameters)}
  \begin{figure}
      \centering
      \includegraphics[width=\textwidth]{../thesis/figures/chap2/fit_omega_inside.png}
  \end{figure}
\end{frame}

\begin{frame}{\textbf{Data analysis}: ACF outside (fits)}
  \hspace{-0.05\textwidth}
  \begin{minipage}{0.47 \textwidth}
    \centering
    \adjincludegraphics[width=\linewidth, trim={{.5\width} 0 0 0}, clip]{../thesis/figures/chap2/fit_size_outside.png}
  \end{minipage}
  \hspace{0.01\textwidth}
  \begin{minipage}{0.50 \textwidth}
      Fit function:
      \[
        \mathcal{A}_{\rm fit}(x) = (1-\Delta)\exp[-\frac{x}{2\ell_1}] + \Delta
      \]
  \end{minipage}
\end{frame}

\begin{frame}{\textbf{Data analysis}: ACF outside (parameters)}
  \begin{figure}
    \centering
    \includegraphics[width = 0.6\textwidth]{../thesis/figures/chap2/param_size_outside.png}
  \end{figure}
\end{frame}

\begin{frame}{\textbf{Data analysis}: ACF outside (parameters)}
  \begin{figure}
      \centering
      \includegraphics[width=\textwidth]{../thesis/figures/chap2/fit_omega_outside.png}
  \end{figure}
\end{frame}

\begin{frame}{\textbf{Conclusions}}
  What did we learn?
  \begin{itemize}
    \item The system shows \textbf{different properties} between inside and outside of the bubble
    \item Border width of the bubble \textbf{depends on the coupling} strength $\Omega_R$
    \item Growth factor of the bubble size in time is \textbf{independent} of $\Omega_R$
    \item In the bubble, periodic structures \textbf{disappear} with size increasing. They \textbf{appear}, instead, outside of the bubble.
    \item Length scale of information outside is related to the Rabi \textbf{healing length}
  \end{itemize}
  \pause
  Future research:
  \begin{itemize}
    \item Analysis of the \textbf{density} channel
    \item Comparison with numerical GPE \textbf{simulations}
    \item Behavior at different \textbf{temperatures}
  \end{itemize}
\end{frame}

\begin{frame}
  \huge
  Thank you for your attention!
\end{frame}

% \backupbegin
% \begin{frame}
  
% \end{frame}
% \backupend

\end{document}
