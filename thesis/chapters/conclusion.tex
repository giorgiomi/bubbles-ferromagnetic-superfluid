In this thesis we characterized bubbles in a ferromagnetic superfluid by analyzing experimental measurements conducted during the first observation of false vacuum decay performed in the Pitaevskii BEC Center laboratories of the University of Trento. 

First, we separated the bubbles into three distinct regions: inside, outside and domain walls. This allowed us to study separately the regions during the bubble evolution. From the raw data we extracted the bubble magnetization shots, obtained by a spin-selective imaging process. We performed two fit routines on the shots, that resulted in acquiring information about the bubble size $\sigma_B$, the length of the inside region and the characteristic width of the domain walls $w_D$. 

After that, we analyzed these parameters by clustering the data and looking at their relative dependence and at their behavior versus the experimental waiting time. By looking at the dependence with the radiation coupling $\Omega_R$, we discovered a non-constant behavior of the average domain wall width $\langle w_D\rangle$, meaning that the spatial extension of the domain wall region depends on $\Omega_R$. We also fitted the bubble size vs time data, obtaining a growth factor value $B$ constant with the coupling, suggesting an equal growing profile for the bubbles in time, independent of $\Omega_R$.

Subsequently, we performed a spectral analysis on the inside and outside regions, by utilizing FFT and ACF. We confronted the two methods and decided to rely on the ACF, since the information given by the FFT was dominated by the signal noise and difficult to analyze. 
The ACF analysis provided information on the system in relation to the bubble size and coupling strength, revealing that the results differed significantly between inside and outside.
In the inside region, the bubble initially presents periodic structures, but with increasing size it loses them, eventually uniforming to a flat magnetization profile. The characteristic length $\ell_1$ of the signal has also a dependence on $\Omega_R$, with a similar behavior of $w_D$. On the contrary, the outside region presents no structures at the beginning but acquires them with size increasing. By comparing the results in both regions, our physical intuition is that, along with its evolution, the system transfers the periodic information from the inside to the outside region. An interesting relation was discovered between the typical length scale of the outside region and the Rabi healing length of the coupled system, and we found that the two quantities are compatible within fit errors.

In conclusion, the results showed that the work done within this thesis provided physically intuitive results and let us explore new physical properties of bubbles in ferromagnetic superfluids by an experimental point of view. 
Future research could focus on incorporating the analysis of the total density channel or further investigating the dependence on the detuning. Another potential direction is to compare the experimental data with numerical GPE simulations. Additionally, exploring the system's behavior at different temperatures, although requiring more measurements, could yield interesting insights.
