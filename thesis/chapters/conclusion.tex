In this thesis we characterized bubbles in a ferromagnetic superfluid by analyzing experimental measurements conducted during the first observation of false vacuum decay performed by the Pitaevskii BEC Center laboratories of the University of Trento. 

First, we separated the bubbles into three distinct regions: inside, outside and border. This allowed us to study separately the regions during the bubble evolution. From the raw data we extracted the bubble magnetization shots, obtained by a spin-selective imaging process. We performed two fit routines on the shots, that resulted in acquiring information about the bubble size $\sigma_B$, the length of the inside region and the characteristic width of the border $w_B$. 

After that, we analyzed these parameters by clustering the data and looking at their relative dependence and at their behaviour with the experimental waiting time. By looking at the dependence with the radiation coupling $\Omega_R$, we discovered a non-constant behaviour of the average border width $\langle w_B\rangle$, meaning that the spatial extension of the border region depends on $\Omega_R$. We also fitted the bubble size vs time data, obteining a growth factor value $B$ constant with the coupling, suggesting an equal growing profile for the bubbles in time, independent of $\Omega_R$.

Subsequently, we performed a spectral analysis on the inside and outside regions, by utilizing FFT and ACF. We confronted the two methods and decided to rely on the ACF, since the information given by the FFT was dominated by the signal noise and difficult to analyze. The ACF analysis provided information on the system in relation to the bubble size and coupling strength. In the inside region, the bubble initially presents periodic structures, but with increasing size it loses them, eventually uniforming to a flat magnetization profile. The periodicity of the signal has a dependence also on $\Omega_R$, with a similar behaviour of $w_B$. On the contrary, the inside region presents no structures at the beginning but acquires them with size increasing. By comparing the results in both regions, our physical intuition is that, along with its evolution, the system transfers the periodic information from the inside to the outside region.

Eventually, this thesis showed that the work done provided good physically intuitive results and let us explore new physical properties of bubbles in ferromagnetic superfluids by an experimental point of view. Further advancements in the research may try to confront the experimental data to numerical simulations and explore the beahviour of the system at different temperatures, or to analyze the density data.
