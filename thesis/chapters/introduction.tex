This thesis originates from the first experimental observation of the false vacuum decay (FVD) via bubble formation phenomenon, made by the Pitaevskii BEC Center laboratories of the University of Trento and presented in Ref.\ \cite{zenesini2024false}. This process is a direct consequence of metastability, namely the finite lifetime of a state when a lower-energy configuration exists but can be reached only through the tunnelling of an energy barrier. FVD is a widely studied phenomenon, explored from a theoretical standpoint in a variety of energy and length scales: from quantum many-body systems to protein-folding and cosmology. However, since in the cosmological case the energy scales are experimentally unavailable, dealing with smaller scales may be the right path to observe and characterize the process.

The framework of the experiment in Trento is of quantum nature and relies on a ferromagnetic superfluid mixture of Bose-Einstein condensates, specifically a system of Sodium atoms optically trapped and cooled below the condensation temperature, forming a two-component spin mixture. The superfluidity of the system guarantees its spatial coherence and thus that it behaves as a single quantum field. The ferromagnetism is a consequence of the double-well energy landscape.

The purpose of this work is to characterize the bubbles formed during the decay process by analyzing the magnetization of the system. The goal is to study the bubble properties such as its location and spatial extension in the condensate in relation to the experimental parameters, controlled and tuned in the laboratory apparatus, but also to the temporal quantities playing a role in the experiment, which will contribute to a study of the evolution of the system. The method used will be the data analysis of the measurements done for the FVD observation experiment.

In the first chapter, a quick theoretical background on the quantum properties of our system will be provided, where we will introduce the condensation phenomenon through the simple case of the ideal Bose gas and then add the interaction component with the Gross-Pitaevskii equation. We will then describe the properties of a two-component mixture alone and in the presence of a coherent coupling between the spieces, that enables ferromagnetism. We will conclude with a brief explanation of how bubbles form and evolve in FVD, along with some experimental remarks.

In the second chapter we will focus instead on the main work of this thesis: the analysis of the experimental data. We will begin by extracting the visible bubble parameters from the images of the system and then proceed with their analysis. Successively, a spectral analysis will be performed on the bubbles for a further exploration of their properties.

Eventually, we will draw conclusions on our work and evaluate if the adopted procedures produced physically reasonable results and contributed to the experimental study of the phenomenon.