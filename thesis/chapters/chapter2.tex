In this chapter we describe the experimental platform in which the measurements were carried out and we then proceed with the characterization of the false vacuum decay bubbles through data analysis. Data is taken from the measurements done for the FVD experiment of Ref.\ \cite{zenesini2024false}. Our aim is to study the magnetization properties of the system in the bubble (\textit{inside} region), around its domain walls (\textit{border} regions) and also in the \textit{outside} region, as seen in Fig.\ \ref{fig:2bubbles}. 
\begin{figure}[ht!]
    \centering
    \includegraphics[width=\textwidth]{figures/chap2/two_bubbles.png}
    \caption{Separation of the system into inside and outside region by left and right borders for two example bubbles. The lower bubble is measured after a higher waiting time and it has a bigger size.}
    \label{fig:2bubbles}
\end{figure}
We are looking for correlations between the external parameters $\Omega_R$ and $\delta_{\rm eff}$, and the bubble properties such as its size $\sigma_B$, its border width $w_B$ and the experimental waiting time $t$. It is also possible to notice some periodic structures in the system, both inside and outside. Then, for a spectral analysis, FFT and ACF will be utilized. 

The code for the analysis was written in the \textbf{\texttt{Python}} programming language. The references used in this chapter are mainly \cite{cominotti2023ferro}, \cite{farolfi2021}, \cite{cominotti2023experiments} and \cite{cominotti2024ultracold}.

\section{Experimental platform}
The experimental platform is composed of a bosonic gas of $N \sim 10^6$ $^{23}$Na atoms, optically trapped and cooled below the condensation temperature, with typical peak densities of $n = 10^{14}$ atoms/\unit{\centi\meter\cubed}. The initial spin state in which the system is prepared is $\ket{F, m_F} = \ket{2, -2} = \ket{\uparrow}$, with $F$ being the total angular momentum of the atom 
% ($\mathbf{F} = \mathbf{I} + \mathbf{J}$, takes into account the nuclear spin and the total angular momentum of the electrons) 
and $m_F$ its projection on the quantization axis. The $\ket{\uparrow}$ state is then coupled to $\ket{1, -1} = \ket{\downarrow}$ through microwave radiation with amplitude $\Omega_R$. The relevant scattering lengths concerning the two states are $a_\uparrow = 64.3 a_0$, $a_\downarrow = 54.5 a_0$ and $a_{\uparrow\downarrow} = 64.3 a_0$.

The trapping potential is harmonic in all three directions, but strongly asymmetric concerning the radial ($\rho$) and axial ($x$) directions. In fact, the trapping frequencies are respectively $\omega_\rho/2\pi = 2\ \unit{\kilo\hertz}$ and $\omega_x/2\pi = 20\ \unit{\hertz}$, yielding an elongated system (cigar-shaped) with inhomogeneous density. The spatial size of the system is given by the Thomas-Fermi radii $R_\rho = 2\ \unit{\micro\meter}$ and $R_x = 200\ \unit{\micro\meter}$, calculated with Eq.\ \eqref{eq:TF}. This particular setup is helpful for suppressing the radial spin dynamics of the condensate and thus reducing it from a 3D system of density $n_{3D}(r_\perp,x)$ to a 1D system of linear density $n(x)$, with the only difference being the renormalization of the $\delta g$ parameter that appears in Eq.\ \eqref{eq:E_MF}:
\[
    \delta g \to k = \frac{2}{3}\frac{n_{3D}}{n}\delta g\, .
\]

In order to extract the density distribution, the two spin states are treated independently one from another, and a spin-selective imaging process is performed. Then, an integration along the transverse direction is carried out, obtaining two 1D density profiles $n_\uparrow(x)$ for to the atoms in the state $\ket{\uparrow}$ and $n_\downarrow(x)$ for the atoms in the state $\ket{\downarrow}$, from which one can extract the relative magnetization
\begin{equation}
    Z(x) = \frac{n_\uparrow(x) - n_\downarrow(x)}{n_\uparrow(x) + n_\downarrow(x)}\, .
    \label{eq:magnetization}
\end{equation}
% Due to the quantum nature of the system, once imaged the condensate loses its spatial information and it has to be prepared back in the initial state, allowing for another measurement. 

\section{Raw data}
Raw data is organized in a hierarchical system. At a fixed instant, the condensate's measured data are called a \textit{shot} (it refers to the imaging process). Each shot is part of a series of them that can be analyzed as the time evolution of a single system: this series is called a \textit{sequence}. Eventually, during a \textit{day} of measurements, many sequences may be collected, and a selection of them will be studied in the following analysis. For each sequence, the experimental data contains also the radiation coupling $\Omega_R$ in a range between $2\pi \times 200$ \unit{\hertz} and $2\pi \times 800$ \unit{\hertz} (it changes from one day of measurements to another) and the detuning $\delta_{\rm eff}$ in a range between $2\pi \times 100$ \unit{\hertz} and $2\pi \times 600$ \unit{\hertz}. For each shot, the experimental waiting time $t$ provides information on the time interval passed before the imaging process, ranging from 1 \unit{\milli\second} to 300 \unit{\milli\second}. 

A shot contains all the information on the system after waiting for a time $t$, including the two population densities, $n_\uparrow(x)$ and $n_\downarrow(x)$, distributed on a length scale from 0 to 400 pixels. The spatial resolution of the image is $1\ \text{pixel} = 1.025\ \unit{\micro\meter} \approx 1\ \unit{\micro\meter}$, so the two length units will often be used interchangeably. The magnetization data $Z(x)$ is calculated with Eq.\ \eqref{eq:magnetization} and, by definition, composed of a series of values ranging from $-1$ (all atoms in the state $\ket{\downarrow}$) to $1$ (all atoms in the state $\ket{\uparrow}$).

Since our focus is to characterize the bubble properties, first we need to extract the spatial parameters of the bubble in order to locate it in each shot, then we will proceed with the study of its evolution.

\subsection{Bubble fit parameters}
\begin{figure}[ht!]
    \centering
    \includegraphics[width = \textwidth]{figures/chap2/arctan_fit.png}
    \caption{Example of fit results performed on a shot. First, the data is fitted with the double-arctangent function of Eq.\ \eqref{eq:double-atan}, then each shoulder is fitted with a single-arctangent, namely the one of Eq.\ \eqref{eq:atan}. This ensures a better estimation of the shoulder centers and thus of the bubble width. Eventually, a last fit is done with the piecewise function of Eq.\ \eqref{eq:fit_exp}, in order to capture the inside/outside discontinuity and the exponential tails.}
    \label{fig:atan-fit}
\end{figure}
In order to localize the bubble in a shot, the most interesting parameters to retrieve are the bubble center $x_0$ and its width $\sigma_B$.
However, not all shots contain a bubble, namely the ones taken when the bubble was not formed yet. We can easily classify the two types of shots by computing the magnetization average in the central region and using a threshold value, set to $Z_{\rm thr} = -0.2$. 

To parametrize the bubble, the magnetization data is fitted with a double-arctangent function
\begin{equation}
    Z_{\rm fit}(x) = -A \left[\frac{2}{\pi}\arctan(\frac{x-c_1}{w_1}) - \frac{2}{\pi}\arctan(\frac{x-c_2}{w_2})\right] + \Delta\, ,
    \label{eq:double-atan}
\end{equation}
where $c_1$ and $c_2$ are the centers' positions of the arctangent "shoulders", and $w_1$ and $w_2$ are their characteristic widths. 
Then, for a better result, a further fit is performed on each shoulder with a single-arctangent function
\begin{equation}
    Z_{\rm fit}(x) = -A \frac{2}{\pi}\arctan(\frac{x-c_{1,2}}{w_{1,2}}) + \Delta\, ,
    \label{eq:atan}
\end{equation}
yielding the shoulder center position $c_{1,2}$. Eventually, we obtain the bubble center position $x_0 = (c_1 + c_2)/2$ and the bubble width $\sigma_B = c_2 - c_1$.
% In some cases, especially when the bubble is narrow, the fitting procedure to optimize the parameters of Eq.\ \eqref{eq:double-atan}'s function fails and we are forced to use a gaussian profile such as
% \begin{equation*}
%     Z_{\rm fit}(x) = - A \exp\left[-\frac{(x-c)^2}{2\sigma^2}\right] + \Delta\, ,
% \end{equation*}
% with $x_0 = c$ being the bubble center and $\sigma_B = \num{2.335}\ \sigma$ its width.

% While this routine is very accurate for determining the shoulder profile and hence the bubble width, the transition between the shoulders and the inside region (the bubble itself) shows no continuity in many shots. In order to capture the entire profile and especially the boundaries of the inside region, we can fit the data with a continous piecewise function made of two exponential tails and a constant value in the middle, such as
While this routine is very accurate for determining the shoulder profile and hence the bubble width, it may be useful to try fitting the data to another function, such as
\begin{equation}
    Z_{\rm fit}(x) = 
    \begin{cases}
        (A-\Delta_1) \exp(\frac{x-x_1}{w_1}) + \Delta_1 \qquad\quad &\text{for } x < x_1\\
        A \qquad\qquad\qquad\qquad &\text{for } x_1 < x < x_2\\
        (A-\Delta_2) \exp(-\frac{x-x_2}{w_2}) + \Delta_2 \qquad &\text{for } x > x_2
    \end{cases}\, ,
    \label{eq:fit_exp}
\end{equation}
which is made of two exponentials and a constant value in the middle. As it turns out, this approach ensures a better estimation of the inside region, whose limits are $x_1$ and $x_2$. The exponential tails give also a characteristic width for the border region, namely $w_B = (w_1 + w_2)/2$.

An example of fitting with the arctangent and stepwise exponential functions is provided in Fig.\ \ref{fig:atan-fit}.

\subsection{Shot sorting}
Once the bubble width $\sigma_B$ is retrieved, it is useful to order the shots in a sequence by this parameter. This process lets us display the system evolution, in contrast to the original shot ordering based on the experimental time waited before observing the bubble. The reason why we are more interested in the size dependence is that the bubble formation event is a stochastic process (the tunneling through the mean-field energy barrier), and the time at which a bubble forms is not determined. However, theory suggests that once the bubble is formed, its evolution (the decay from FV to TV) is determined. 
An example of sorting by both time and size is shown in Fig.\ \ref{fig:sorting} with a colormap displaying the magnetization profiles (blue is for positive $Z$ and red for negative $Z$).

\begin{figure}[ht!]
    \centering
    \includegraphics[width=\textwidth]{figures/chap2/shot_sorting.png}
    \caption{Example of initial shot sorting based on experimental waiting time (on the left) and final sorting based on bubble width $\sigma_B$ (on the right) for all shots with $\Omega_R/2\pi = 400$ \unit{\hertz} and $\delta_{\rm eff}/2\pi = -596.5$ \unit{\hertz}. The $\sigma_B$ parameter is estimated from the previous fitting procedure, and the no-bubble shots ($\sigma_B = 0$) are removed from the right plot. The data is colored by mapping positive magnetization to blue and negative magnetization to red.}
    \label{fig:sorting}
\end{figure}

\section{Parameters analysis}
\label{sec:params}
\begin{figure}[ht!]
    \centering
    \includegraphics[width = \textwidth]{figures/chap2/clustering.png}
    \caption{Data clustering with the K-Means algorithm and $n_{\rm clusters} = 20$ based only on $t$ (first two columns from the left) or $\sigma_B$ (third column). The first column shows the $\sigma_B$ data on the y-axis, while the last two show the $w_B$ data.}
    \label{fig:clust}
\end{figure} 
In order to look at the distribution of the bubble parameters versus the experimental time or the bubble size, we decide to group all shots by the radiation coupling $\Omega_R$ (ignoring the detuning $\delta_{\rm eff}$) and plot the bubble width $\sigma_B$ and border width $w_B$. 

An approach to this might be the one of ordering the shots by time or size and then dividing all shots in blocks with a fixed number of shots per block. However, the distributions of $t$ and $\sigma_B$ are not continuous (especially $t$, which is discretized), and it may be convenient to look for clustered data. 
The clustering approach can be implemented by the \texttt{\textbf{K-Means}} algorithm from the \texttt{\textbf{Scikit-Learn}} Python library, which lets us choose the number of clusters. The clustered data, with $n_{\rm clusters} = 20$, is shown in Fig.\ \ref{fig:clust}. 
\begin{figure}[ht!]
    \centering
    \includegraphics[width = \textwidth]{figures/chap2/b_param_cluster.png}
    \caption{Bubble parameters distribution averaged on all shots of the same cluster. The upper panel shows, from left to right, $w_B$ vs $\sigma_B$, $w_B$ vs $t$ and $\sigma_B$ vs $t$ (the last one in log-log scale). Errors are estimated with the standard deviation of the values in the cluster. The lower panel presents on the left the average border width over all shots with the same coupling radiation $\Omega_R$. A linear fit on the log-log increasing data is performed in order to estimate the rate of growth, shown on the right of the bottom panel.}
    \label{fig:b_param}
\end{figure}

The bubble parameters, averaged in each cluster and in relation one to another, are shown in Fig.\ \ref{fig:b_param}. The border width does not show any visible pattern when varying the bubble size, and for this reason it is averaged and shown as a function of $\Omega_R/2\pi$ in the upper right plot, where it increases with $\Omega_R$ although the points at 400 Hz and 600 Hz interrupt the monotony. The behavior against time is slightly increasing for small time and decreasing for long times, but the most significant trend is clearly the one of $\sigma_B$ vs $t$, where the bubble size is increasing with time at first, and then decreases. The initial behavior corresponds to the power-law growth of $\sigma_B$ in time of the type
\[
    \sigma_B(t) = A\left(\frac{t}{1\ \unit{\milli\second}}\right)^B\, , 
\]
the final one is probability an artifact due to the low-density tails of the Thomas-Fermi distribution. Eventually, a linear fit is performed on the log-log data with the aim of finding the power-law coefficient $B$ for each value of the radiation coupling, shown in the lower right corner of Fig.\ \ref{fig:b_param}. The values are all compatible among them, hence revealing that the bubble grows with the same profile at different values of $\Omega_R$.

\section{Spectral analysis}
Since we are interested in the properties of the bubble throughout its evolution, periodic signals in the magnetization channel and their relation to the waiting time or the bubble size are important features to focus on. In order to study them, a spectral profile is much needed, from which one can extrapolate the main frequencies of the signals and, most importantly, the typical length scales of the system.

The tools used in the following analysis are the Fast Fourier Transform (FFT) and the autocorrelation function (ACF).

\subsection{FFT and ACF definition}
\paragraph{FFT}
We will first approach the problem of deriving such a profile by using the Fast Fourier Transform (FFT), an algorithm that implements the Discrete Fourier Transform (DFT) in an efficient manner.\footnote{There are several efficient methods to compute the DFT. The most common implementation is the Cooley-Tukey algorithm, used in the \texttt{\textbf{Scipy}} Python library for FFT.} Given the input as a sequence of $N$ discrete values $Z_0,\dots,Z_{N-1}$ sampled with spacing $\Delta x$, by definition the DFT is a series of $N$ discrete values $\mathcal{F}_0,\dots,\mathcal{F}_{N-1}$ spaced by $\Delta k = 1/(N\Delta x)$ and convoluted with a complex phase such that
\begin{equation*}
    \mathcal{F}_k = \sum_{n=0}^{N-1} Z_n e^{-2\pi i \frac{k}{N}n}\, .
\end{equation*}
When the input $Z_n$ is real-valued, the transform is too, and it is also symmetric between positive and negative frequencies. The physical world contains only positive frequencies, so we will neglect the negative part of the transform. This is achievable by using the Scipy function \texttt{\textbf{rfft}} instead of \texttt{\textbf{fft}}.

The result of the Fourier transform of a signal is the extraction of the main frequency components of the signal itself. In fact, as we know, the transform of a pure sinusoidal signal with frequency $\omega_0$ results in a Dirac delta at $\omega = \omega_0$ (excluding the negative frequencies). In the general case of a signal made up of more frequency components and with noise, the peaks in the Fourier transform will correspond the most relevant frequencies. Note that in our case the signal is in the space-domain, and by taking the inverse of the points in the frequency-domain, one can get information about the spatial periodicity of the system, hence the typical length scale.

\paragraph{ACF}
Another tool that can be used to study the periodic properties of a signal is the autocorrelation function (ACF). Similarly to the DFT, the ACF is also a particular type of convolution, where the signal is convoluted with itself. The caveat here is that since our signal is of finite length, the whole convolution would show boundary effects, as shown in Fig.\ \ref{fig:ACF_b+w}.
\begin{figure}[ht!]
    \centering
    \includegraphics[width=\textwidth]{figures/chap2/ACF_b+w.jpeg}
    \caption{Schematic representation of ACF. Boundary effects arise when computing the autocorrelation function on the whole signal (top). Since the signal is finite, the contributions on the borders are set to zero. When restricting to windowed autocorrelation, no boundary effects are present anymore (bottom).}
    \label{fig:ACF_b+w}
\end{figure}
It is then worth limiting the signal in a central window of length $2W$ and computing the ACF on the windowed signal (Fig.\ \ref{fig:ACF_b+w}). Taking as input the latter as $Z_0,\dots,Z_{2W-1}$, the output will be a series of values $\mathcal{A}_0,\dots,\mathcal{A}_{W}$ living in the space-domain\footnote{While the FFT domain is made of frequencies, the ACF domain is instead made of lag values, corresponding to the spatial shifts of the signal.} such that:
\begin{equation*}
    \mathcal{A}_k = \frac{1}{2} \left( \frac{\sum_{n} Z_n Z_{n+k}}{\sqrt{\sum_{n} Z_n^2 \sum_{n} Z_{n+k}^2}} + \frac{\sum_{n} Z_n Z_{n-k}}{\sqrt{\sum_{n} Z_n^2 \sum_{n} Z_{n-k}^2}} \right)\, .
\end{equation*}
This formula computes the windowed autocorrelation by shifting the signal both to the left and to the right and then taking the average.
Both terms are normalized in order to get $\mathcal{A}_0 = 1$. 
Note that the sums run from $n = 0$ to $n = W-1$ and thus the length of the signal must be greater than $4W$: data that does not respect this condition will not be analyzed.

The computation of this function on a simple sinusoidal signal is shown in App.\ \ref{chap:app_ACF}. In contrast to the FFT, the ACF results, as mentioned before, live in the space-domain and give direct results on the length scales. A high ACF value at $\Delta x_0$ means that the signal is correlated with itself when shifted by $\Delta x_0$. The periodicity of the signal is then related to the length $\Delta x_0$.

\subsection{FFT and ACF analysis on sequences}
Now, what we shall do is computing the FFT and the ACF on the data, taking care of the fact that it is necessary to separate the inside region (the bubble) from the outside one. This is done by relying on the piecewise exponential fit results, namely the positions $x_1$ and $x_2$ from Eq.\ \eqref{eq:fit_exp}.
\begin{figure}[ht!]
    \centering
    \includegraphics[width=\textwidth]{figures/chap2/inside_omdet.png}
    \caption{Example of FFT and ACF calculated on the inside region of shots with $\Omega_R/2\pi = 400$ \unit{\hertz} and $\delta_{\rm eff}/2\pi = -596.5$ \unit{\hertz}, after selecting the shots where $x_2-x_1 > 4W = 80$ \unit{\micro\meter}. The values for each shot are shown in the left graphs with colormaps, while the averages on all shots are on the right. Note that before computing the transforms the data was set to zero-mean by subtracting its average. In the lower right graph, ACF computed on the true data (without the 0-mean) is shown for comparison to the other ACF profile.}
    \label{fig:inside_00}
\end{figure}
% We will also compute the transforms on the no-bubble shots. 
An example of the inside region analysis for a selected pair of values $\Omega_R$, $\delta_{\rm eff}$ (the same of Fig.\ \ref{fig:sorting}) is presented in Fig.\ \ref{fig:inside_00}, where the data was first set to zero-mean and only the shots where $x_2-x_1 > 4W$ are selected, with $W = 20$ \unit{\micro\meter}.
% One can probably (with a good eye) already see in the colormap plot that two or three vertical bands extend over all shots and appear at the same frequencies. This ensures that the FFT averaging is rightful.

Let us briefly discuss the behaviors of the computed FFTs and ACFs, considering that it is similar for all $\Omega_R$, $\delta_{\rm eff}$ and the example presented is a good one (see also Fig.\ \ref{fig:inside_avg}, showing the profiles averaged over all sequences with the same $\Omega_R$). The FFT shows a peak at a frequency in the order of $k_{\rm FFT} \sim 0.01$ \unit{\per\micro\meter} and of width in the order of $\Delta k_{\rm FFT} \sim 0.1$ \unit{\per\micro\meter}. The ACF instead has a first peak at $\Delta x_{\rm ACF} \sim 10-11$ \unit{\micro\meter}. By comparing these results and relying on the FFT and ACF definitions, the relation should be $k = 1/\Delta x$. However, by taking the inverse of the ACF peak values, one gets $k_{\rm ACF} \sim 0.1$ \unit{\per\micro\meter}, a frequency hidden in the FFT plot due to the broad peak. The reason for the peak broadness is probability the data noise, which is difficult to analyze properly. We will thus proceed by relying only on the ACF for the remaining analysis.

\begin{figure}[t!]
    \centering
    \includegraphics[width=\textwidth]{figures/chap2/inside_fft_avg.png}
    \caption{FFT and ACF profiles (computed with zero-mean data) averaged over all sequences with the same radiation coupling $\Omega_R$. On the top, the FFTs are plotted both in the $k$-domain and $x$-domain. On the bottom, che ACFs are also plotted in the two domains to match the upper panel and confront the results.
    The FFTs are peaked at $k_{\rm FFT} \sim 0.01$ \unit{\per\micro\meter}, while the first ACF peaks are at $\Delta x_{\rm ACF} \sim 10-11$ \unit{\micro\meter}.}
    \label{fig:inside_avg}
\end{figure}

\subsection{ACF analysis on all shots}
From now on, all shots will be considered, taking into account all $\Omega_R$ and $\delta_{\rm eff}$ values. What we shall proceed to do is studying the ACF parameters as functions of the bubble properties $t$ and $\sigma_B$, and of the experimental external parameters $\Omega_R$ and $\delta_{\rm eff}$.
The routine is the following:
\begin{enumerate}
    \item Gather all shots with the same $\Omega_R$ and save their bubble parameters;
    \item Select only the shots where the inside or outside region is at least $4W$ pixels long;
    \item Sort the shots based on the waiting time or the bubble size;
    \item Cluster the shots based on the sorting parameter (as in Sec.\ \ref{sec:params});
    \item Compute the ACF profile for each shot in a cluster and take the average;
    \item Plot the ACF averages for each cluster and fit them;
    \item Plot the fit results vs the time or size averages in each cluster.
\end{enumerate}
The number of analyzed inside and outside shots is shown in Tab.\ \ref{tab:shots}.
\begin{table}[ht!]
    \centering
    \begin{tabular}{c|c|c|c}
        $\Omega_R/2\pi$ [\unit{\hertz}] & \# all shots & \# inside shots & \# outside shots \\
        \hline
        300 & 1491 & 1435 & 938\\   
        400 & 660 & 604 & 489\\
        600 & 718 & 561 & 717\\
        800 & 159 & 117 & 159\\     
    \end{tabular}
    \caption{Number of shots selected for the ACF computation. The inside shots must have $x_2-x_1 > 4W$, while the outside ones must have both $x_1 > 4W$ and $400\ \unit{\micro\meter} - x_2 > 4W$. It is noticeable that the inside data for 300 \unit{\hertz} is way bigger than the rest.}
    \label{tab:shots}
\end{table}
Let us now discuss the results by starting from the inside analysis and then proceeding with the outside one.

\paragraph{Inside}
The ACF outside profiles can be fitted with a damped cosine of the form
\begin{equation}
    \mathcal{A}_{\rm fit}(x) = (1 - \Delta)\cos(\frac{\pi x}{\ell_2})\exp[-\frac{1}{2}\left(\frac{x}{\ell_1}\right)^\alpha] + \Delta\, ,
    \label{eq:damp}
\end{equation}
with the fit parameters being $\ell_1$, $\ell_2$ and $\Delta$. The $\ell_1$ parameter provides information on the length scale of the damping due to noise or, in other words, it gives a characteristic length at which the signal loses its information on being periodic. $\ell_2$ indicates instead the length scale of the spatial patterns in the signal, as shown in App.\ \ref{chap:app_ACF}. The offset $\Delta$ is a measure of how much the signal is uniformed, in fact in the limit case of a constant signal, $\Delta = 1$ and $\mathcal{A}(x) = 1\ \forall x$.
Lastly, the exponent $\alpha = 1.7$ was fine-tuned in order to capture the profile in the best way possible.\footnote{Literature suggests to use $\alpha = 2$, a gaussian damping.} One could theoretically set $\alpha$ as a fit parameter, but this would result in a worse estimation of the length $\ell_1$, with the fit that adapts to the data by changing only $\alpha$. 

\begin{figure}[ht!]
    \centering
    \includegraphics[width = \textwidth]{figures/chap2/fit_size_inside.png}
    \caption{ACF average profiles of inside shots clustered by size in 9 blocks (solid lines) and fitted with the damped cosine of Eq.\ \eqref{eq:damp} (dotted lines) for each value of $\Omega_R$. Fits are performed until $\Delta x = 11$ \unit{\micro\meter} except at 300 Hz. Size increases from darker to lighter colors.}
    \label{fig:fit_size_inside}
\end{figure}
\begin{figure}[ht!]
    \centering
    \includegraphics[width = \textwidth]{figures/chap2/param_size_inside.png}
    \caption{ACF fit parameters $\ell_1$, $\ell_2$ and $\Delta$ of inside shots clustered by size. The $\ell_2$ data when the fit fails is not shown.}
    \label{fig:param_size_inside}
\end{figure}
\begin{figure}[ht!]
    \centering
    \includegraphics[width = \textwidth]{figures/chap2/param_time_inside.png}
    \caption{ACF fit parameters $\ell_1$, $\ell_2$ and $\Delta$ of inside shots clustered by time (in log scale). The $\ell_2$ data when the fit fails is not shown.}
    \label{fig:param_time_inside}
\end{figure}
\begin{figure}[ht!]
    \centering
    \includegraphics[width = \textwidth]{figures/chap2/fit_omega_inside.png}
    \caption{On the first plot from the left, ACF average profiles of inside shots grouped by $\Omega_R$ (solid lines) and fitted with the damped cosine (dotted lines). Errors are estimated with the standard deviation of the ACF values. On the other plots, fit parameters $\ell_1$, $\ell_2$ and $\Delta$ vs $\Omega_R$.}
    \label{fig:fit_omega_inside}
\end{figure}
An example of fit on the ACF average profiles of the inside shots is presented in Fig.\ \ref{fig:fit_size_inside}, where the shots are clustered by size. Due to the noise of profiles with a little number of shots, except from $\Omega_R/2\pi = 300$ \unit{\hertz} the fits are performed until $\Delta x = 11$ \unit{\micro\meter}. Sometimes the estimation of $\ell_2$ fails, it is then set to $\ell_2 = 12$ \unit{\micro\meter} and the fit is repeated for the other parameters.

The parameters retrieved from the fits are in Fig.\ \ref{fig:param_size_inside}, along with the ones from the fits on time-clustered data in \ref{fig:param_time_inside}. 
The size dependence seems to be the most noticeable. The offset $\Delta$ is clearly increasing (note the clean data of 300 Hz, where the number of shots is bigger) and approaching the value of 1. As mentioned before, this means that when the bubble grows in size its magnetization becomes more uniformed, the system progressively loses local information and its periodic structures disappear.
The length $\ell_1$ is constant and bigger than the spin healing length $\xi_s \approx 0.8$ \unit{\micro\meter} of Eq.\ \eqref{eq:heal}. 
The length $\ell_2$ does not show a clear trend, due to the many failed fits.

Eventually, a global fit on all shots with the same radiation coupling is made and shown in Fig.\ \ref{fig:fit_omega_inside}. The $\ell_2$ plot is not shown because its estimation succeeds only for $\Omega_R/2\pi = 300$ \unit{\hertz}. Here it is interesting to notice the behavior of $\Delta$ as a function of $\Omega_R$, because it reminds of the average border width $\langle w_B \rangle$ in Fig.\ \ref{fig:b_param}, especially when looking at the strange 400 Hz and 600 Hz points. The length $\ell_1$ does not show a noticeable behavior.

\paragraph{Outside}
For the outside region ACF fit we prefer to use a decreasing exponential
\begin{equation}
    \mathcal{A}_{\rm fit}(x) = (1-\Delta)\exp[-\frac{x}{2\ell_1}] + \Delta\, .
    \label{eq:exp}
\end{equation}
The reason is that the outside region does not show any particular patterns and the ACF profiles are very noisy.
An example of fit (performed until $\Delta x = 11$ \unit{\micro\meter}) on the outside shots is presented in Fig.\ \ref{fig:fit_size_outside}, where the shots are clustered by size as in Fig.\ \ref{fig:fit_size_inside}. The parameters retrieved from the fit are in Fig.\ \ref{fig:param_size_outside}, along with those from the time-clustered data in \ref{fig:param_time_outside}. 
\begin{figure}[ht!]
    \centering
    \includegraphics[width = \textwidth]{figures/chap2/fit_size_outside.png}
    \caption{ACF average profiles of outside shots clustered by size in 9 blocks (solid lines) and fitted with the exponential of Eq.\ \eqref{eq:exp} (dotted lines) for each value of $\Omega_R$. Fits are performed until $\Delta x = 11$ \unit{\micro\meter}. Size increases from darker to lighter colors.}
    \label{fig:fit_size_outside}
\end{figure}
\begin{figure}[ht!]
    \centering
    \begin{minipage}[t]{0.47 \textwidth}
        \centering
        \includegraphics[width = \textwidth]{figures/chap2/param_size_outside.png}
        \caption{ACF fit parameters $\ell_1$ and $\Delta$ of outside shots clustered by size.}
        \label{fig:param_size_outside}
    \end{minipage}
    \hspace{0.02\textwidth}
    \begin{minipage}[t]{0.47 \textwidth}
        \centering
        \includegraphics[width = \textwidth]{figures/chap2/param_time_outside.png}
        \caption{ACF fit parameters $\ell_1$ and $\Delta$ of outside shots clustered by time (in log scale).}
        \label{fig:param_time_outside}
    \end{minipage}
\end{figure}
\begin{figure}[ht!]
    \centering
    \includegraphics[width = \textwidth]{figures/chap2/fit_omega_outside.png}
    \caption{On the first plot from the left, ACF average profiles of outside shots grouped by $\Omega_R$ (solid lines) and fitted with the damped cosine (dotted lines) until $\Delta x = 11$ \unit{\micro\meter}. On the other plots, fit parameters $\ell_1$ vs $1/\sqrt{\Omega_R}$ and $\Delta$ vs $\Omega_R$. In the central plot, the Rabi healing length $\xi_R$ is shown as a function of the coupling.}
    \label{fig:fit_omega_outside}
\end{figure}
What emerges outside of the bubble is that the offset $\Delta$ is decreasing with size, while $\ell_1$ remains constant like in the inside. The decrease of $\Delta$ is the exact opposite behavior of the inside region, where the offset increases with the bubble size. The meaning of this is that outside of the bubble structures begin to appear and it is possible that the system during the bubble expansion acquires information that was previously stored inside.

A global fit on all shots with the same radiation coupling is in Fig.\ \ref{fig:fit_omega_outside}. Here, we plot the length $\ell_1$ against $1/\sqrt{\Omega_R}$ and we observe that the two quantities are correlated. Furthermore, by looking at Eq.\ \eqref{eq:heal}, it is the Rabi healing length $\xi_R$ that scales in this manner, and after plotting it we see a clear correspondence between $\xi_R$ and $\ell_1$, hence we can suggest that the system, outside of the bubble, has a typical length of information equal to its Rabi healing length.

% \subsection{Border alignment}
