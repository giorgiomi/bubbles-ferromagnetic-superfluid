
\Universita{University of Trento}
\Dipartimento{Department of Physics}
\CorsoDiLaurea{Bachelor's Degree in Physics }
\AnnoAccademico{Academic Year 2023--2024}
\Titolo{Bubbles in a ferromagnetic superfluid}
\Relatore{Dr. Alessandro \textsc{Zenesini}}
\RelatoreLabel{Supervisor}
\CandidatoLabel{Undergraduate Student}

\Candidato{Giorgio \textsc{Micaglio}} 
\Matricola{227051}
\DataEsame{March 10, 2025}
\Logo{layout/logo_rosso.png}
\LogoWidth{4.5cm} %optional, default: 3cm
\LogoPosition{top}
\LogoSfondo{layout/logo_bn2.png}
\opacitaSfondo{0.05}

\begin{titlepage}
    \newgeometry{left=3cm, right=3cm, bottom=2cm, top =3cm} 
    \pagestyle{empty}
    \makefrontpage
    \restoregeometry
\end{titlepage}

\thispagestyle{empty} % Add this line to suppress the page number

\frontmatter
% dedica
% ~ \newpage
\null\vspace{\stretch{1}}
\begin{flushright}
    \textit{A mamma e papà}
\end{flushright}
\vspace{\stretch{2}}\null

%ringraziamenti
\chapter*{Acknowledgments}
I would first like to express my deepest gratitude to my parents for encouraging me to pursue my academic path and supporting me throughout this journey.

A heartfelt thank you goes to all the friends I met here in Trento: the members of the \textit{AISF local committee}, \textit{Gli Scalatori di Povo}, and my \textit{Lunedì} roommates.

A special thanks goes to my dear friends from \textit{Stasssse}—Anna, Caterina, Federico, Gabriele, Giacomo, Giulia, Matteo, Roberto—with whom I shared some of the most unforgettable moments of my Bachelor years, memories for which I will always be grateful.

I would also like to thank my supervisor, dr.\ Alessandro Zenesini, for providing me with the idea and data for this thesis, as well as for his invaluable guidance and support in developing the methods used for the analysis.

\begin{abstract}
    This thesis investigates the characteristic features of bubbles in a ferromagnetic superfluid, which form by false vacuum decay. The primary objective is to analyze the experimental data to identify and quantify the key properties of these bubbles. By examining the relationship between the experimental parameters and the observed bubble characteristics, this work aims to provide a deeper understanding of the underlying physical phenomena and contribute to the broader fields of ferromagnetism and superfluidity.
\end{abstract}

\tableofcontents

% \printglossaries
% \addcontentsline{toc}{chapter}{Glossary}

% \printnomenclature
% \addcontentsline{toc}{chapter}{Nomenclature list}

\mainmatter
\chapter*{Introduction}
\markboth{\MakeUppercase{Introduction}}{}
\addcontentsline{toc}{chapter}{Introduction}
This thesis originates from the first experimental observation of the false vacuum decay (FVD) via bubble formation phenomenon, made in the Pitaevskii BEC Center laboratories of the University of Trento and presented in Ref.\ \cite{zenesini2024false}. This process is a direct consequence of metastability, namely the finite lifetime of a state when a lower-energy configuration exists but can be reached only through the tunnelling of an energy barrier. FVD is a widely studied phenomenon, explored from a theoretical standpoint in a variety of energy and length scales: from quantum many-body systems to protein-folding and cosmology. However, since in the cosmological case the energy scales are experimentally unavailable, dealing with smaller scales may be the right path to observe and characterize the process.

The framework of the experiment in Trento is of quantum nature and relies on a ferromagnetic superfluid mixture of Bose-Einstein condensates, specifically a system of Sodium atoms optically trapped and cooled below the condensation temperature, forming a two-component spin mixture. The superfluidity of the system guarantees its spatial coherence and thus that it behaves as a single quantum field. The ferromagnetism is a consequence of the double-well energy landscape.

The purpose of this work is to characterize the bubbles formed during the decay process by analyzing the magnetization of the system. The goal is to study the bubble properties such as its location and spatial extension in the condensate in relation to the experimental parameters, controlled and tuned in the laboratory apparatus, but also to the evolution of the system. 

In the first chapter, a quick theoretical background on the quantum properties of our system will be provided, where we will introduce the condensation phenomenon through the simple case of the ideal Bose gas and then add the interaction component with the Gross-Pitaevskii equation. We will then describe the properties of a two-component mixture alone and in the presence of a coherent coupling between the spieces, that enables ferromagnetism. We will conclude with a brief explanation of how bubbles form and evolve in FVD, along with some experimental remarks.

In the second chapter we will focus instead on the main work of this thesis: the analysis of the experimental data. We will begin by extracting the visible bubble parameters from the images of the system and then proceed with their analysis. Successively, a spectral analysis will be performed on the bubbles for a further exploration of their properties.

Eventually, we will draw conclusions on our work and evaluate if the adopted procedures produced physically reasonable results and contributed to the experimental study of the phenomenon.