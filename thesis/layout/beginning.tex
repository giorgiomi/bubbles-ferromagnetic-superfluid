
\Universita{University of Trento}
\Dipartimento{Department of Physics}
\CorsoDiLaurea{Bachelor's Degree in Physics }
\AnnoAccademico{Academic Year 2023--2024}
\Titolo{Bubbles in a ferromagnetic superfluid}
\Relatore{Dr. Alessandro \textsc{Zenesini}}
\RelatoreLabel{Supervisor}
\CandidatoLabel{Graduate Student}

\Candidato{Giorgio \textsc{Micaglio}} 
\Matricola{227051}
\DataEsame{March 10, 2025}
\Logo{layout/logo_rosso.png}
\LogoWidth{4.5cm} %optional, default: 3cm
\LogoPosition{top}
\LogoSfondo{layout/logo_bn2.png}
\opacitaSfondo{0.05}

\begin{titlepage}
    \newgeometry{left=3cm, right=3cm, bottom=2cm, top =3cm} 
    \pagestyle{empty}
    \makefrontpage
    \restoregeometry
\end{titlepage}

\thispagestyle{empty} % Add this line to suppress the page number

\frontmatter
% dedica
% ~ \newpage
\null\vspace{\stretch{1}}
\begin{flushright}
    \textit{A mamma e papà}
\end{flushright}
\vspace{\stretch{2}}\null

%ringraziamenti
\chapter*{Acknowledgments}
I would first like to express my deepest gratitude to my parents for encouraging me to pursue my academic path and supporting me throughout this journey.

A heartfelt thank you goes to all the friends I met here in Trento: the members of the \textit{AISF local committee}, \textit{Gli Scalatori di Povo}, and my \textit{Lunedì} roommates.

A special thanks goes to my dear friends from \textit{Stasssse}—Anna, Caterina, Federico, Gabriele, Giacomo, Giulia, Matteo, Roberto—with whom I shared some of the most unforgettable moments of my Bachelor years, memories for which I will always be grateful.

I would also like to thank my supervisor, dr.\ Alessandro Zenesini, for providing me with the idea and data for this thesis, as well as for his invaluable guidance and support in developing the methods used for the analysis.

\begin{abstract}
    This thesis investigates the characteristic features of bubbles in a ferromagnetic superfluid, which form by false vacuum decay. The primary objective is to analyze the experimental data to identify and quantify the key properties of these bubbles. By examining the relationship between the experimental parameters and the observed bubble characteristics, this work aims to provide a deeper understanding of the underlying physical phenomena and contribute to the broader fields of ferromagnetism and superfluidity.
\end{abstract}

\tableofcontents

% \printglossaries
% \addcontentsline{toc}{chapter}{Glossary}

% \printnomenclature
% \addcontentsline{toc}{chapter}{Nomenclature list}

\mainmatter
\chapter*{Introduction}
\markboth{\MakeUppercase{Introduction}}{}
\addcontentsline{toc}{chapter}{Introduction}
This thesis originates from the first experimental observation of False Vacuum Decay (FVD), made by the Pitaevskii BEC Center laboratories of the University of Trento and presented in Ref.\ \cite{zenesini2024false}.