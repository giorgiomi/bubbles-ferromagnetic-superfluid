% general
\usepackage[utf8]{inputenc}
\usepackage{geometry}
\usepackage[english]{babel}
\usepackage{csquotes}
\usepackage{graphicx}
\usepackage{listings}
\usepackage[svgnames]{xcolor}
\usepackage{comment}
\usepackage{setspace}
\usepackage{appendix}
\usepackage{lipsum}     %testo di prova (si può cancellare una volta iniziato a scrivere)
\usepackage[hidelinks]{hyperref}

% font
\usepackage[T1]{fontenc}
\usepackage{palatino}
\usepackage{mathpazo}
% \lstset{keywordstyle={\bfseries \color{blue}}}

%\usepackage[9-15]{pagesel}
% attenzione: la filigrana in prima pagina ha opacità massima con questa opzione attivata
% non mettendo la prima pagina comunque appare draft in background

% math
\usepackage{amsmath}
\usepackage{amssymb}
\usepackage{physics}
\usepackage{braket}
\usepackage{siunitx}

% first page
\usepackage{frontespizio}
\usepackage{tikz}
\usepackage[pages=some]{background}

% line on top of pages
\usepackage{fancyhdr}
\pagestyle{fancy}
\fancyhf{}
\fancyhead[LE]{\leftmark}
\fancyhead[RO]{\rightmark}
\fancyfoot[LE,RO]{\thepage}


% bibliography
\usepackage[style=numeric,sorting=none,sortcites=true]{biblatex}
\addbibresource{biblio.bib}

% abstract
\newenvironment{abstract}%
    {\cleardoublepage%
        \thispagestyle{empty}%
        \null \vfill\begin{center}%
            \bfseries \abstractname \end{center}}%
        {\vfill\null}

% analytic index
\usepackage{imakeidx}
\makeindex

% symbols and notation
\usepackage{nomencl}
\makenomenclature

% to upload as last packet in order to avoid problem
\usepackage{glossaries}
% \makeglossaries
\newglossaryentry{set}{name={set},description={a collection of objects}}
\newglossaryentry{due}{name={second},description={second example}}
\newglossaryentry{emptyset}{name={\ensuremath{\emptyset}},description={the empty set}}
\longnewglossaryentry{fishage}{name={Fish Age}}{
    A common name for the Devonian geologic period spanning from the end of the Silurian Period to the beginning of the Carboniferous Period.
    
    This age was known for its remarkable variety of fish species.
}
\newglossaryentry{elite}{name={{é}lite},description={select group or class}}

